\section{Introduction}\label{sec:1}

\noindent 1.1\hspace{1em}In an elegant paper \cite{serrin_symmetry_1971},
Serrin considered solutions of second order elliptic equations satisfying 
over-determined boundary conditions. For equations with spherical symmetry he proved that the 
domain on which the solution is defined is necessarily a ball and that the solution is spherically 
symmetric. The proof is based on the maximum principle and on a device (which goes back to 
Alexandroff; see Chap.~7 in \cite{hopf_differential_2003}) of moving parallel planes 
to a critical position and then showing 
that the solution is symmetric about the limiting plane.

In this paper we will use the same technique to prove symmetry of positive solutions of elliptic 
equations vanishing on the boundary---as well as related results (including some extensions to 
parabolic equations). Some of the equations we treat are related to physics and our techniques 
should be applicable to other physical situations. We study solutions in bounded domains and in 
the entire space. The simplest example in a bounded domain is

\begin{theorem}\label{thm:1}
  In the ball $\Omega: |x|<R$ in $\mathbb{R}^n$, let $u>0$ be a positive solution
  in $C^2(\overline{\Omega})$ of
  \begin{equation}\label{eq:1.1}
    \Delta u + f(u) = 0 \quad \text{with } u=0 \text{ on } |x|=R. 
  \end{equation}
  Here $f$ is of class $C^1$. Then $u$ is radially symmetric and
  \[\frac{\partial u}{\partial r} < 0,\quad \text{for } 0<r<R.\]
\end{theorem}

The point of interest is that the result holds no matter what $f$ is.
We should note that $f(u)\geq 0$ for all $u$,
implies that any nontrivial solution is automatically positive in $\Omega$.

Theorem \ref{thm:1} is a special case of Theorem \ref{thm:1prime} in Sect.~\ref{sec:2}.
Another simple special result is:

\begin{theorem}\label{thm:2}
  Let $u>0$ be a $C^2$ solution of \eqref{eq:1.1} in a ring-shaped domain
  \[R' < |x| \leq R.\]
  Then
  \begin{equation}
    \frac{\partial u}{\partial r} < 0 \quad \text{for } \frac{R'+R}{2}\leq |x| < R.
  \end{equation}
\end{theorem}

This implies that $u$ can have no critical point in this larger half of the ring. 
Note that no condition is imposed at the boundary $|x|=R'$.

A third simple result (see Theorem \ref{thm:3prime} in Sect.~\ref{sec:3}
for a more general form) is:

\begin{theorem}\label{thm:3}
  Let $u(x)$ be a $C^2$ solution of the ordinary differential equation
  \begin{equation}\label{eq:1.3}
    u'' + b(x)u' + f(u) = 0\quad \text{on} \quad 0<x<1,
  \end{equation}
  with $u$ continuous on $0<x\leq 1$ and $u(x)>u(1)$ for $0<x<1$.
  Here $f\in C^1$, and $b(x)$ is continuous in $0<x<1$.
  If $b(x)\geq 0$ everywhere then
  \begin{equation}\label{eq:1.4}
    u' < 0\quad \text{on} \quad \frac12 < x < 1.
  \end{equation}
  Furthermore if $u'(\frac12) = 0$ then $u$ is symmetric about $\frac12$
  and $b(x)$ is necessarily identically zero.
\end{theorem}

As an example: $u = 1 - \cos 2\pi x$ is a solution of
\[u'' + 4\pi^2(u-1) = 0, \quad 0\leq x\leq 1\]
satisfying all the conditions of the theorem.

\medskip

\noindent 1.2\hspace{1em}Our interest in these questions grew out of a study of positive solutions
in $\mathbb{R}^n$, $n > 2$ ($n = 4$ in particular) of the equation
\begin{equation}\label{eq:1.5}
  \Delta u + u^{\frac{n+2}{n-2}} = 0.
\end{equation}
This is the Euler equation for the function (``action'')
\[A(u) = \int_{\mathbb{R}^n} \biggl(\frac{1}{2}|\nabla u|^2 - \frac{u^\beta}{\beta}\biggr)\,dx,
  \quad \beta = \frac{2n}{n-2}.\]
Equation \eqref{eq:1.5} and the associated action are conformally invariant 
in the sense that if $u$ is a solution, then after a conformal mapping $x\mapsto y$ the function
\begin{equation}
  v(y) = u(x) J^{\frac{2-n}{2n}}(x),
\end{equation}
where $J$ is the Jacobian, is also a solution. For $n=4$, Eq.~\eqref{eq:1.5}
as well as the corresponding, but simpler, equation $\Delta u = u^{\frac{n+2}{n-2}}$
in \cite{loewner_partial_1974},
and the linear equation $u = 0$, give rise to bona-fide solutions of the classical Euclidean
Yang-Mills equations via `t Hooft's Ansatz \cite{corrigan_scalar_1977,wilczek_geometry_1977}. 
Equation~\eqref{eq:1.5} has the explicit solutions (replace $x$ by $ix$ in the solutions of 
page 250 of \cite{loewner_partial_1974})
\begin{equation}\label{eq:1.7}
  u(x) = \frac{[n(n-2)\lambda^2]^{\frac{n-2}{4}}}{\bigl(\lambda^2+|x-x_0|^2\bigr)^{n/2-1}}
\end{equation}
for $\lambda>0$, $x_0\in \mathbb{R}^n$.
These solutions yield the well-known one-instanton solutions in a regular gauge of the Yang-Mills 
equations. We used the same methods as in the proofs of the preceding results to show that these
are the only positive solutions in $\mathbb{R}^n$ with reasonable behaviour at infinity,
namely $u=O(|x|^{2-n})$. This behaviour follows from the finiteness of the action---as was 
proved by K.~Uhlenbeck (private communication). This uniqueness result together with the well 
known properties (see \cite{loewner_partial_1974}) of $\Delta u = u^{(n+2)/(n-2)}$ and 
of $\Delta u = 0$ (\cite{hopf_differential_2003}) show 
that any finite action solution of the full Yang-Mills equations given by `t Hooft's Ansatz is 
self-dual.

Subsequently it was brought to our attention by R.~Schoen that our uniqueness result 
is equivalent to the following geometric result due to Obata \cite{obata_conjectures_1971}:
A Riemannian metric on $S^n$ which is conformal to the usual one and having the same constant scalar curvature,
is in fact the pullback of the usual one under a conformal map of $S^n$ to itself.

As a demonstration of the use of our methods in the full space we present a related result for 
equations which are rotationally, but not necessarily conformally, invariant.
For convenience we suppose $n>2$.

\begin{theorem}\label{thm:4}
  Let $v>0$ be a $C^2$ solution of an elliptic equation
  \begin{equation}\label{eq:1.8}
    F(v, |\nabla v|^2, \sum v_jv_kv_{jk}, \tr A, \tr A^2, \ldots, \tr A^n) = 0
    \quad\text{in}\quad \mathbb{R}^n
  \end{equation}
  where $A$ is the Hessian matrix $\{v_{ij}\}$, here $F\in C^1$ for $v>0$
  and all values of the other arguments.

  Assume that near infinity, $v$ and its first derivatives admit the asymptotic
  expansion (using summation convention):
  \begin{equation}\label{eq:1.9}
    \begin{aligned}
      v & = \frac{1}{|x|^m}\biggl(a_0 + \frac{a_jx_j}{|x|^2} + \frac{a_{jk}x_jx_k}{|x|^4}
            + o\biggl(\frac{1}{|x|^2}\biggr)\biggr) \\
      v_{x_i} & = -\frac{m}{|x|^{m+2}} x_i\biggl(a_0 + \frac{a_jx_j}{|x|^2}\biggr)
            + \frac{a_i}{|x|^{m+2}} - \frac{2x_i}{|x|^{m+4}}a_jx_j
            + O\biggl(\frac{1}{|x|^{m+3}}\biggr)
    \end{aligned}
  \end{equation}
  for some $m, a_0> 0$. Then $v$ is rotationally symmetric about some point and $v_r<0$ 
  for $r > 0$ where $r$ is the radial coordinate about that point.
\end{theorem}

Here, to say that an equation $G(x, v, v_i, v_{jk})= 0$ is elliptic means 
$G_{v_{jk}}$ is a positive definite matrix for all values of the arguments.

It is natural to ask under what conditions one can assert that expansions of the 
form~\eqref{eq:1.9} hold. Here is an example:
\begin{proposition}\label{prop:1}
  Let $v\in C^{2+\mu}$, $0<\mu<1$, be a positive solution of
  \begin{equation}\label{eq:1.8prime}
    \Delta v + f(v) = 0 \tag{$1.8'$}
  \end{equation}
  with $v(x) = O(|x|^{2-n})$ as $x\to\infty$. Assume that for some
  $k\geq\frac{n+2}{n-2}$, $g(v) = f(v)v^{-k}$ is H\"older continuous on $0\leq v\leq v_0$
  where $v_0$ is the maximum of our solution. Then expansions~\eqref{eq:1.9},
  with $m=n-2$, hold for $|x|$ large.
\end{proposition}

Example. The function
\[v(x) = (1+|x|^4)^{-(n-2)/4}\]
satisfies~\eqref{eq:1.8prime} with
\[f(v) = (n^2-4) v^{\frac{n+6}{n-2}} \biggl(v^{\frac{4}{2-n}}-1\biggr)^{1/2}.\]
This satisfies the conditions of the proposition with $k = \frac{n+4}{n-2}$.

\medskip

\noindent 1.3\hspace{1em}Our results are based on a well-known form of the maximum principle \cite{protter_maximum_1984} for
a $C^2$ solution $u\leq 0$ of the differential inequality (we use summation convention)
\begin{equation}\label{eq:1.10}
  Lu\equiv a_{ij}(x)u_{x_ix_j} + b_i(x)u_{x_i} + c(x)u \geq 0
\end{equation}
in a domain $\Omega$ (open connected subset of $\mathbb{R}^n$),
and the corresponding Hopf boundary lemma \cite{protter_maximum_1984}. 
Here $L$ is a uniformly elliptic operator, i.e., for some constant $c_0>0$
\begin{equation}\label{eq:1.11}
  a_{ij}\xi_i\xi_j \geq c_0|\xi|^2,
\end{equation}
and the coefficients of $L$ are uniformly bounded in absolute value.

\begin{MP}
If $u$ vanishes at some point in $\Omega$ then $u\equiv 0$.
\end{MP}

We use the Hopf lemma in the form:

\begin{lemmaH}
  Suppose there is a ball $B$ in $\Omega$ with a point $P\in\partial\Omega$
  on its boundary and suppose $u$ is continuous in $\Omega\cup P$ and $u(P)=0$.
  Then if $u\not\equiv 0$ in $B$ we have for an outward directional derivative at $P$,
  \[\frac{\partial u}{\partial \nu}(P) > 0,\]
  in the sense that if $Q$ approaches $P$ in $B$ along a radius then
  \[\lim_{Q\to P} \frac{u(P)-u(Q)}{|P-Q|} > 0.\]
\end{lemmaH}

This is well known in case $c(x)\leq 0$ (see Theorem 7, p.~65 of \cite{protter_maximum_1984}) but,
as already observed by Serrin in \cite{serrin_symmetry_1971} p.~310, the more general result follows by the same
argument used to prove the maximum principle in the form above.
For the convenience of the reader we include the derivation, using the well known
result for case $c\equiv 0$.

\begin{proof}
  With
  \[v = e^{-\alpha x_1} u, \quad \alpha>0\]
  one obtains
  \[0\leq Lu = e^{\alpha x_1} L'v + vL(e^{\alpha x_1})\]
  where $L'$ is an elliptic operator containing no zero-order term. Thus
  \[0\leq L'v + v(a_{11}\alpha^2 + b_1\alpha +c) = L'v + c'v.\]
  For $\alpha$ sufficiently large, $c'\geq 0$. Hence, since $v\leq 0$,
  \[L'v \geq 0\quad \text{in } \Omega.\]
  As $v(P)=0$ we have by the usual form of the Hopf lemma,
  \[\frac{\partial v}{\partial \nu}(P)>0,\]
  and the desired result follows since $u_{\nu}(P) = e^{\alpha x_1} v_{\nu}(P)$.
\end{proof}

We shall also use the following consequence of the maximum principle.

\begin{corollary*}
  Suppose in \eqref{eq:1.10} that for some positive constants $m,b,c_1$
  \[a_{11}\geq m>0,\quad b_1\geq -b,\quad c\leq c_1.\]
  Assume that $\overline{\Omega}$ lies in a narrow region
  \[a-\varepsilon < x_1 < a,\]
  and $u$ is a solution of the inequality \eqref{eq:1.10} in $\Omega$
  with $u\leq 0$ on $\partial\Omega$. Then $u\leq 0$ in $\Omega$ provided
  \[c_1\exp(2b\varepsilon/m) \leq c_1 + 2b^2/m.\]
\end{corollary*}

The proof makes use of arguments in \cite{courant_methods_1989} pages 330--331;
for convenience we present it here. See also \cite{protter_maximum_1984} pages 73--74.

\begin{proof}
  For $\alpha = 2b/m$ the function
  \[g = e^{\alpha a} - e^{\alpha x_1}\]
  is positive in $\overline{\Omega}$ and satisfies
  \begin{align*}
    -Lg
    & = (a_{11}\alpha^2 + b_1\alpha)e^{\alpha x_1} -c(e^{\alpha a}-e^{\alpha x_1}) \\
    & \geq (m\alpha^2-b\alpha)e^{\alpha x_1} - c_1(e^{\alpha a}-e^{\alpha x_1}).
  \end{align*}
  Thus
  \begin{align*}
    -e^{-\alpha x_1} Lg
    & \geq m\alpha^2 - b\alpha + c_1 - c_1 e^{\alpha\varepsilon} \\
    & = \frac{2b^2}{m} + c_1 - c_1 e^{\alpha\varepsilon}\quad (\text{since } \alpha=2b/m) \\
    & \geq 0\quad \text{by our hypothesis}.
  \end{align*}
  Consequently the function
  \[v = \frac{u}{g}\]
  satisfies
  \[L'v + v\frac{Lg}{g} \geq 0\quad\text{in } \Omega\]
  where $L'$ is an elliptic operator with no zero-order term.
  Since $Lg/g\leq 0$ and $v\leq 0$ on $\partial\Omega$,
  the usual form of the maximum principle yields $v\leq 0$,
  and hence $u\leq 0$ in $\Omega$.
\end{proof}

  
In Sect.~\ref{sec:3} we will also make use of an extension of the Hopf boundary lemma,
at a corner, due to Serrin (Lemma 2 in \cite{serrin_symmetry_1971}). Since it may be of further interest
we present a slightly more general form (requiring however a bit more smoothness of
the coefficients):

\begin{lemmaS}
  Let $\Omega$ be a domain in $\mathbb{R}^n$ with the origin $O$ on its boundary.
  Assume that near $O$ the boundary consist of two transversally intersecting
  $C^2$ hypersurfaces $\varrho = 0$ and $\sigma = 0$.
  Suppose $\varrho,\sigma<0$ in $\Omega$. Let $w$ be a function in $C^2(\overline{\Omega})$,
  with $w<0$ in $\Omega$, $w(0)=0$, satisfying the differential inequality \eqref{eq:1.10}
  in $\Omega$ with uniformly bounded coefficients satisfying \eqref{eq:1.11}. Assume
  \begin{equation}\label{eq:1.12}
    a_{ij}\varrho_{x_i}\sigma_{x_j} \geq 0\quad \text{at } 0.
  \end{equation}
  If this is zero assume furthermore that $a_{ij}\in C^2$ in $\overline{\Omega}$
  near $O$, and that
  \begin{equation}\label{eq:1.12prime}
    D(a_{ij}\varrho_{x_i}\sigma_{x_j}) = 0 \quad \text{at } O, \tag{$1.12'$}
  \end{equation}
  for any first order derivative $D$ at $O$ tangent to the submanifold
  $\{\varrho=0\}\cap \{\sigma=0\}$.
  Then, for any direction $s$ at $O$ which enters $\Omega$ transversally to each hypersurface,
  \begin{equation}\label{eq:1.13}
    \begin{aligned}
      & \frac{\partial w}{\partial s} < 0 \quad\text{at } O 
        \text{ in case of strict inequality in \eqref{eq:1.12}}, \\
      & \frac{\partial w}{\partial s} < 0\quad \text{or} \quad
        \frac{\partial^2w}{\partial s^2} < 0\quad
        \text{at } O \text{ in case of equality in \eqref{eq:1.12}}
    \end{aligned}
  \end{equation}
\end{lemmaS}

Note that conditions \eqref{eq:1.12} and \eqref{eq:1.12prime} are invariant under
change of coordinates, and of the choices of the particular functions $\varrho$ 
and $\sigma$ representing the bounding hypersurfaces.
The proof will be presented in the Appendix together with a rough extension
in case \eqref{eq:1.12} is weakened (see Lemmas \ref{lem:A.1} and \ref{lem:A.2}).

\medskip

\noindent 1.4\hspace{1em}In addition to the maximum principle and Lemmas H and S
we use the procedure of moving up planes perpendicular to a fixed direction as in \cite{serrin_symmetry_1971},
and we shall now describe it geometrically.

In the following $\Omega$ is a bounded domain in $\mathbb{R}^n$ with smooth boundary.
(Some domains with corners will be discussed in \S 3.)

Let $\gamma$ be a unit vector in $\mathbb{R}^n$ and let $T_{\lambda}$ be the
hyperplane $\gamma\cdot x = \lambda$. For $\lambda=\tilde{\lambda}$ large,
$T$ is disjoint from $\overline{\Omega}$.
Let the plane move continuously toward $\Omega$,
preserving the same normal, i.e., decrease $\lambda$,
until it begins to intersect $\overline{\Omega}$.
From that moment on, at every stage the plane $T_{\lambda}$ will cut off from $\Omega$
an open cap $\Sigma(\lambda)$, the part of $\Omega$ on
the same side of $T_{\lambda}$ as $T_{\tilde{\lambda}}$.
Let $\Sigma'(\lambda)$ denote the reflection of $\Sigma(\lambda)$ in the plane $T_{\lambda}$.
At the beginning, $\Sigma'(\lambda)$ will be in $\Omega$ and as $\lambda$ decreases,
the reflected cap $\Sigma'(\lambda)$ will remain in $\Omega$,
at least until one of the following occurs:

\begin{enumerate}[(i)]
  \item $\Sigma'(\lambda)$ becomes internally tangent to $\partial\Omega$ at some point $P$
    not on $T_\lambda$ or
  \item $T_\lambda$ reaches a position where it is orthogonal to the boundary of $\Omega$
    at some point $Q$.
\end{enumerate}

We denote by $T_{\lambda_1}: \gamma\cdot x = \lambda_1$ the plane $T_\lambda$
when it first reaches one of these positions and we call $\Sigma(\lambda_1)=\Sigma_{\gamma}$
the maximal cap associated with $\gamma$. Note that its reflection $\Sigma_{\gamma}'$
in $T_{\lambda_1}$ lies in $\Omega$.

One of our main results will be that if $u > 0$ is a solution of an elliptic equation
in $\Omega$ satisfying certain conditions, with $u = 0$ on $\partial\Omega$,
then $\gamma\cdot\nabla u<0$ in $\Sigma_\gamma$.

It may be that if we decrease $\lambda$ below $\lambda_1$
the reflection $\Sigma'(\lambda)$ of $\Sigma(\lambda)$ in the plane $T_\lambda$
continuous to belong to $\Omega$ as in the following example:

\begin{figure}[htbp]
  \centering
  \includegraphics[width=.7\textwidth]{figure/fig1.pdf}
  \caption{}
  \label{fig:1}
\end{figure}

In that case $\Sigma'(\lambda)$ will remain in $\Omega$ for $\lambda$ in a maximal interval
\[\lambda_2\leq\lambda<\infty,\qquad \lambda_2\leq\lambda_1.\]
We will call the cap $\Sigma(\lambda_2)$ the optimal cap corresponding to the direction $\gamma$. 
Clearly either $\Sigma'(\lambda_2)$ is internally tangent to $\partial\Omega$ 
at some point not on $T_{\lambda_2}$, or $T_{\lambda_2}$ is orthogonal to
$\partial\Omega$ at some point.

A word on notation: For a set $S$ in $\partial\Omega$, by a neighbourhood of $S$ in $\Omega$
we mean $\Omega\,\cap$\,(an open neighbourhood of $S$ in $\mathbb{R}^n$).

Section \ref{sec:2} contains the main results for general second order elliptic equations---including the proofs of Theorem \ref{thm:1}--\ref{thm:3}.
Some extensions are given in 3 and straightforward extensions to 
parabolic equations are briefly described in Sect.~\ref{sec:5}. 
Section \ref{sec:4} is concerned with the proofs of Theorem \ref{thm:4}
and Proposition \ref{prop:1}.
This may be read independently of the other sections.
The Appendix contains the proof of Lemma S.
